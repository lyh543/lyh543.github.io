% LaTeX 常见语法

% 以下是引入宏包,可以理解为 C 语言的 #include<stdio.h>
% 直接复制到新的 LaTeX 文档就行
% 如果使用了网上的语法而报错:Undifined ...,可能是没有引入对应的宏包,可百度该语法对应的宏包
\documentclass{article}
\usepackage{amsmath}
\usepackage{amssymb}
\usepackage{enumerate}
\usepackage{graphics}
\usepackage{graphicx}
\usepackage{adjustbox}
\usepackage{listings}
\usepackage{bm}
\usepackage{booktabs}
\usepackage[UTF8]{ctex}

% 在 \begin{document} 和 \end{document} 的内容将被输出到 pdf
\begin{document}

\section{标题}

\subsection{这是一个小标题}

\subsubsection{这是一个更小的标题}

% 没有 subsubsub 了

\section{文段}

这是一段文字,注意有首行缩进。
一个回车不能换行。

两个回车可以换行,并且新行有首行缩进。\\
而两个反斜线和一个回车的换行没有首行缩进。\\

两个反斜线和两个回车的换行有首行缩进,并且有空行。\\

另外,习惯上,中文文段中使用中文标点符号(如逗号、括号)。汉字和数字/英语之间间隔一个空格。中文标点符号和数字/英语之间不用间隔空格。如:\\

错误示范:当BMI小于25时,认为不肥胖.当BMI大于25 ,认为肥胖。

正确示范:当 BMI 小于 25 时,认为不肥胖。当BMI大于 25,认为肥胖。\\

不过,对于公式的引用,由于被引用的公式的形式为

\begin{equation}
a+b=c
\end{equation}

引用公式时,为与原公式保持一致,即使在中文文段中,也使用英文括号,且在小括号的两侧添加空格。如 (1) 公式。

\section{列表}

\begin{enumerate}
	\item 增加体内的肌肉。体内的肌肉越多,代谢率越高。
	\item 将有氧运动和无氧运动结合,让身体得到全方位锻炼。
	\item 某些食物,如绿茶,能短时间增加基础代谢率。
	\item 早餐要吃好。长期不吃早餐会导致人体新陈代谢速率降低。
	\item 少食多餐。两餐之间相隔较长时间会减缓新陈代谢。
\end{enumerate}

还可以这样:

\begin{enumerate}[(1)]
	\item 增加体内的肌肉。体内的肌肉越多,代谢率越高。
	\item 将有氧运动和无氧运动结合,让身体得到全方位锻炼。
	\item 某些食物,如绿茶,能短时间增加基础代谢率。
	\item 早餐要吃好。长期不吃早餐会导致人体新陈代谢速率降低。
	\item 少食多餐。两餐之间相隔较长时间会减缓新陈代谢。
\end{enumerate}

\section{表格}

表格推荐使用在 https://www.tablesgenerator.com/ 自动生成。可将 Default table style 更换为 Bookstab table style 以获得三线表格。

\begin{table}[!htbp]
	\caption{肥胖影响因素调查分析结果}\label{tab:obese_reason} \centering
	% \caption 是该表的标题
	% \label 是该表的内部名称
	% \centering 将该表居中
	\begin{tabular}{llll}
		\toprule
		& \textbf{肥胖患病率}(\%) & \textbf{卡方值} & \textbf{P值}\\
		\midrule
		\textbf{性别}     &      &        &                 \\
		~~男      & 13.2 & 169.8  & \textless{}0.01 \\
		\ \ 女      & 8.1  &        &                 \\
		\textbf{城乡}     &      &        &                 \\
		~~ 城镇     & 10.0 & 18.4   & \textless{}0.01 \\
		~~ 农村     & 11.8 &        &                 \\
		\textbf{学历}     &      &        &                 \\
		~~ 小学及以下  & 18.1 & 200.1  & \textless{}0.01 \\
		~~ 中学     & 10.8 &        &                 \\
		~~ 大学     & 8.3  &        &                 \\
		\textbf{生活压力感}  &      &        &                 \\
		~~ 从没有    & 13.6 & 49.3   & \textless{}0.01 \\
		~~ 偶尔     & 10.3 &        &                 \\
		~~ 经常     & 9.5  &        &                 \\
		\textbf{睡眠时间}   &      &        &                 \\
		~~ 不足 6 h & 13.7 & 28.5   & \textless{}0.01 \\
		~~ 6 h 以上 & 10.2 &        &                 \\
		\textbf{体质等级}   &      &        &                 \\
		~~ 不合格    & 28.3 & 1425.4 & \textless{}0.01 \\
		~~ 合格     & 13.7 &        &                 \\
		~~ 良好     & 5.4  &        &                 \\
		~~ 优秀     & 2.4  &        &                 \\
		\bottomrule
	\end{tabular}
\end{table}

\section{图片}

下图将引用与 tex 文件同目录的 recommend\_plan.pdf/png/jpg 图片文件。

\begin{figure}[!htbp]
	\centering
	\includegraphics[width=.6\textwidth]{recommend_plan}
	\caption{腰臀比与 BMI 的拟合图}
	\label{fig:whr-bmi}
\end{figure}

\section{公式}

有三种引用公式的方法:\\

行内公式:$a+b=c$\\

带标号的行间公式:

\begin{equation}
a^2 + b^2 = c^2
\label{gougu}
\end{equation}

不带标号的行间公式:

$$a^n+b^n=c^n \ \  n \geq 3$$

由于公式的知识实在太多,可以单写一篇文章,这里不再赘述。只介绍常用的符号。

$$a + \frac{b}{c} - d \times e \div f \leq g \geq h$$

$$\int_{0}^{2} xdx = 2$$

$$\min f_1 = \sum_{i=1}^{11} A_{i1}x_i$$

$$x_i \geq \begin{cases}
1 & i = 1, 5, 6, 7, 8\\
0 & Others
\end{cases}$$

$$\alpha \ \beta \ \theta \ \gamma \ \Delta$$

% 空格可使用 $\$ 长空格可使用 $\quad$

\section{图、表、公式的引用}

在上文,我们给图、表定义了 label。这是一种内部名称,只在 tex 文档里可见。

这样的好处是实际的图 1、2、3 的编号交给 LaTeX,作者仅需用自己喜欢的名称对图片进行命名,然后在正文引用对应的名称即可。当图片的顺序发生变化时(如图 1 前面新插入了一张图片,此时图 1、2、3 应顺延为图 2、3、4),LaTeX 仍能保证正确的引用关系,无需作者修改。

需要注意的是,LaTeX 由于是从上到下进行编译,第一次编译的时候无法将引用正确地链接到原文,出现 [??] 的情况。因此,如果被引用的东西发生了变化,请进行\textbf{两次编译}。

下面是例子:\\

图 \ref{fig:whr-bmi} 展示了腰臀比和 BMI 的拟合图。\\

公式 (\ref{gougu}) 被称为勾股定理。\\

LaTeX 会自动为该引用生成对应的数字。

\section{参考文献的引用}

和图、表等的引用相同,我们需要给每一个参考文献定义一个“内部名字”,LaTeX 在编译时会自动给参考文献编号,并连接原文。\\

张三说的对。\cite{Zhang3}

李华说的也对。\cite{Li10}

\begin{thebibliography}{}
	
	\bibitem{Li10}
	L. Ming, Y. Shucheng, R. Kui, and L. Wenjing, Securing Personal Health Records in Cloud Computing: Patient-Centric and Fine-Grained
	Data Access Control in Multi-owner Settings, in: Processing of SecureComm 2010, LNICST 50, pp. 89-106, 2010.
	
	\bibitem{Zhang3}
	R. Zhang, and L. Liu, Security Models and Requirements for Healthcare Application Clouds, in: Processing of  Cloud 2010, pp. 268-275, 2010.
	
\end{thebibliography}

\section{代码}

默认模板生成的可能不好看,可以考虑使用别的模板。

\begin{lstlisting}[language=matlab]
A = ElementInFood(3:end,:);
b = ElementNeed(3:end);
f = ElementInFood(1,:);
lb = [1 0.5 0.5 0.5 0.5 0.5 0.5 0.5 0.25 0.25 0.25];
ub = 7 * ones(1,11);
[x2,fval2,exitflag,output] = linprog(f,-A,-b,[],[],lb,ub);
lb = [1 0 0 0 1 1 1 1 0 0 0];
ub = 10 * ones(1,11);
A = [A; 0 1 1 1 0 0 0 0 0 0 0];
b = [b, 1];
[x1,fval1,exitflag,output] = linprog(f,-A,-b,[],[],lb,ub);
\end{lstlisting}


\end{document}